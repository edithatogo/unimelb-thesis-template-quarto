\documentclass{article}

\title{A Modernised Thesis Template}
\author{Doughnut}
\date{2025-08-03}

\begin{document}

\maketitle

\bookmarksetup{startatroot}

\chapter{A Modernised Thesis
Template}\label{a-modernised-thesis-template}

This is the main index file for the Quarto thesis template. The content
of the thesis is organized into chapters and appendices, which are
defined in \texttt{\_quarto.yml}.

See the \texttt{README.md} for instructions on how to use this template.

\bookmarksetup{startatroot}

\chapter{Abstract}\label{abstract}

This is where you should write your thesis abstract. It is typically a
concise summary of your work, often limited to one page.

\bookmarksetup{startatroot}

\chapter{Declaration of Authorship}\label{declaration-of-authorship}

I, Doughnut, declare that this thesis titled, ``A Modernised Thesis
Template'' and the work presented in it are my own. I confirm that:

\begin{itemize}
\item
  The thesis comprises only my original work towards the
  \textbf{?meta:thesis.degree} except where indicated in the preface;
\item
  due acknowledgement has been made in the text to all other material
  used; and
\item
  the thesis is fewer than the maximum word limit in length, exclusive
  of tables, maps, bibliographies and appendices as approved by the
  Research Higher Degrees Committee.
\end{itemize}

Signed:

\begin{center}\rule{0.5\linewidth}{0.5pt}\end{center}

Date:

2025-08-03T00:00:00+00:00

\bookmarksetup{startatroot}

\chapter{Preface}\label{preface}

Where applicable, the following information must be included in a
preface: * a description of work towards the thesis that was carried out
in collaboration with others, indicating the nature and proportion of
the contribution of others and in general terms the portions of the work
which the student claims as original; * a description of work towards
the thesis that has been submitted for other qualifications; * a
description of work towards the thesis that was carried out prior to
enrolment in the degree; * whether any third party editorial assistance
was provided in preparation of the thesis and whether the persons
providing this assistance are knowledgeable in the academic discipline
of the thesis; * the contributions of all persons involved in any
multi-authored publications or articles in preparation included in the
thesis; * the publication status of all chapters presented in article
format using the descriptors below; * Unpublished material not submitted
for publication * Submitted for publication to {[}publication name{]} on
{[}date{]} * In revision following peer review by {[}publication name{]}
* Accepted for publication by {[}publication name{]} on {[}date{]} *
Published by {[}publication name{]} on {[}date{]} * an acknowledgement
of all sources of funding, including grant identification numbers where
applicable and Australian Government Research Training Program
Scholarships, including fee offset scholarships.

\bookmarksetup{startatroot}

\chapter{Acknowledgements}\label{acknowledgements}

The acknowledgements and the people to thank go here, don't forget to
include your project advisor\ldots{}

\bookmarksetup{startatroot}

\chapter{Glossary}\label{glossary}

\begin{description}
\tightlist
\item[LaTeX]
A document preparation system
\item[SVM]
Support Vector Machine
\end{description}

\bookmarksetup{startatroot}

\chapter{Abbreviations}\label{abbreviations}

\begin{longtable}[]{@{}ll@{}}
\toprule\noalign{}
Acronym & Meaning \\
\midrule\noalign{}
\endhead
\bottomrule\noalign{}
\endlastfoot
LAH & List Abbreviations \\
\end{longtable}

\bookmarksetup{startatroot}

\chapter{Constants}\label{constants}

\begin{longtable}[]{@{}
  >{\raggedright\arraybackslash}p{(\linewidth - 4\tabcolsep) * \real{0.2059}}
  >{\centering\arraybackslash}p{(\linewidth - 4\tabcolsep) * \real{0.0882}}
  >{\raggedright\arraybackslash}p{(\linewidth - 4\tabcolsep) * \real{0.7059}}@{}}
\toprule\noalign{}
\begin{minipage}[b]{\linewidth}\raggedright
Constant Name
\end{minipage} & \begin{minipage}[b]{\linewidth}\centering
Symbol
\end{minipage} & \begin{minipage}[b]{\linewidth}\raggedright
Value
\end{minipage} \\
\midrule\noalign{}
\endhead
\bottomrule\noalign{}
\endlastfoot
Speed of Light & \(c\) &
\(2.997\ 924\ 58\times10^{8}\ \mbox{ms}^{-\mbox{s}}\) (exact) \\
\end{longtable}

\bookmarksetup{startatroot}

\chapter{Symbols}\label{symbols}

\begin{longtable}[]{@{}lll@{}}
\toprule\noalign{}
Symbol & Name & Unit \\
\midrule\noalign{}
\endhead
\bottomrule\noalign{}
\endlastfoot
\(a\) & distance & m \\
\(P\) & power & W (Js\(^{-1}\)) \\
\(\omega\) & angular frequency & rads\(^{-1}\) \\
\end{longtable}

\bookmarksetup{startatroot}

\chapter{Introduction}\label{ch-introduction}

This chapter serves as an introduction to the thesis template and
demonstrates various features, including citations, figures, tables,
equations, and code listings. The structure of this document is based on
the University of Melbourne's guidelines for thesis preparation. Key
works in the field of document preparation include those by Knuth (Knuth
1984) and Lamport (Lamport 1986). We use LaTeX to typeset this document.
We also use a Support Vector Machine (SVM). Further resources can be
found on the Comprehensive TeX Archive Network (CTAN) ({``CTAN -- the
Comprehensive TeX Archive Network''} n.d.).

\section{Citations and Bibliography}\label{sec-citations}

This template uses \texttt{biblatex} for bibliography management, which
offers great flexibility. Here are some examples of citation commands:

\begin{itemize}
\tightlist
\item
  \texttt{@key}: Basic citation, e.g., (Hawthorn, Weber, and Scholten
  2001).
\item
  Parenthetical citation, e.g., (Wieman and Hollberg 1991).
\item
  Textual citation, e.g., Arnold, Wilson, and Boshier (1998) showed
  that\ldots{}
\item
  Context-aware citation, e.g., (Hao 2023).
\item
  Multiple citations: (Hawthorn, Weber, and Scholten 2001; Wieman and
  Hollberg 1991).
\item
  Citations with page numbers: See (Knuth 1984, A:15) for details.
\item
  Another example: Lamport (1986, see Chapter 3) provides an overview.
\end{itemize}

The bibliography is automatically generated at the end of the document.

\section{Figures and Subfigures}\label{sec-figures}

Figures are crucial in a thesis. This template uses the
\texttt{graphicx} package for including images and \texttt{subcaption}
for handling subfigures. Always use vector graphics (like PDF or EPS)
when possible for diagrams and plots to ensure scalability. For
photographs, high-resolution raster graphics (PNG, JPG) are appropriate.

\begin{figure}

\centering{

\pandocbounded{\includegraphics[keepaspectratio]{chapters/Chapter1_files/figure-pdf/fig-tikz_example-1.pdf}}

}

\caption{\label{fig-tikz_example}Example of a figure created with
\texttt{tikz}.}

\end{figure}%

Figure~\ref{fig-tikz_example} shows a simple figure. For more complex
layouts, subfigures can be used, as shown in
Figure~\ref{fig-subfigures_example}.

\begin{figure}

\begin{minipage}{0.50\linewidth}

\centering{

\pandocbounded{\includegraphics[keepaspectratio]{chapters/figures/unimelb_logo.eps}}

}

\subcaption{\label{fig-subfigure_a}First subfigure.}

\end{minipage}%
%
\begin{minipage}{0.50\linewidth}

\centering{

\pandocbounded{\includegraphics[keepaspectratio]{chapters/figures/unimelb_logo.eps}}

}

\subcaption{\label{fig-subfigure_b}Second subfigure.}

\end{minipage}%

\caption{\label{fig-subfigures_example}Example of two subfigures.}

\end{figure}%

Refer to subfigure Figure~\ref{fig-subfigure_a} or
Figure~\ref{fig-subfigure_b} when discussing specific parts.

\section{Tables}\label{sec-tables}

The \texttt{booktabs} package is included for creating
professional-quality tables. It encourages avoiding vertical rules and
using horizontal rules sparingly. Table~\ref{tbl-example_table} provides
an illustration.

\begin{longtable}[]{@{}lll@{}}
\caption{An example table.}\label{tbl-example_table}\tabularnewline
\toprule\noalign{}
Item & Value 1 (V) & Value 2 (A) \\
\midrule\noalign{}
\endfirsthead
\toprule\noalign{}
Item & Value 1 (V) & Value 2 (A) \\
\midrule\noalign{}
\endhead
\bottomrule\noalign{}
\endlastfoot
Measurement 1 & 123.45 & 67.8 \\
Measurement 2 & 1.23 & 45.6 \\
Measurement 3 & 9.0 & 7.8 \\
\end{longtable}

\section{Mathematical Equations}\label{sec-equations}

The \texttt{amsmath} package provides numerous environments for
typesetting mathematics. Here's a simple numbered equation, Equation
Equation~\ref{eq-einstein}:

\begin{equation}\phantomsection\label{eq-einstein}{ E = mc^2 }\end{equation}

For a set of aligned equations, use the \texttt{align} environment, like
in Equations \textbf{?@eq-align\_start} and \textbf{?@eq-align\_end}:

\begin{equation}\phantomsection\label{eq-align}{
\begin{align}
    a + b &= c \label{eq:align_start} \\
    x + y + z &= w \label{eq:align_end}
\end{align}
}\end{equation}

The \texttt{gather} environment can be used for centering multiple
equations:

\begin{equation}\phantomsection\label{eq-gather}{
\begin{gather}
    \int_0^\infty e^{-x^2} dx = \frac{\sqrt{\pi}}{2} \label{eq:gaussian_integral} \\
    \sum_{n=1}^\infty \frac{1}{n^2} = \frac{\pi^2}{6} \label{eq:basel_problem}
\end{gather}
}\end{equation}

Refer to (\textbf{ch-introduction?}) for more examples, or a specific
Section~\ref{sec-equations}.

\section{Code Listings}\label{sec-code}

Quarto supports code listings with syntax highlighting.
\textbf{?@lst-python\_example} shows a Python example.

\begin{Shaded}
\begin{Highlighting}[]
\KeywordTok{def}\NormalTok{ greet(name):}
    \CommentTok{"""This function greets the person passed in as a parameter."""}
    \BuiltInTok{print}\NormalTok{(}\SpecialStringTok{f"Hello, }\SpecialCharTok{\{}\NormalTok{name}\SpecialCharTok{\}}\SpecialStringTok{!"}\NormalTok{)}

\CommentTok{\# Example usage}
\NormalTok{greet(}\StringTok{"World"}\NormalTok{)}
\end{Highlighting}
\end{Shaded}

\{\#lst-python\_example caption=``A simple Python function.''\}

This concludes the demonstration of common thesis elements.

\cleardoublepage
\phantomsection
\addcontentsline{toc}{part}{Appendices}
\appendix

\chapter{An Appendix}\label{an-appendix}

Lorem ipsum dolor sit amet, consectetur adipiscing elit. Vivamus at
pulvinar nisi. Phasellus hendrerit, diam placerat interdum iaculis,
mauris justo cursus risus, in viverra purus eros at ligula. Ut metus
justo, consequat a tristique posuere, laoreet nec nibh. Etiam et
scelerisque mauris. Phasellus vel massa magna. Ut non neque id tortor
pharetra bibendum vitae sit amet nisi. Duis nec quam quam, sed euismod
justo. Pellentesque eu tellus vitae ante tempus malesuada. Nunc
accumsan, quam in congue consequat, lectus lectus dapibus erat, id
aliquet urna neque at massa. Nulla facilisi. Morbi ullamcorper eleifend
posuere. Donec libero leo, faucibus nec bibendum at, mattis et urna.
Proin consectetur, nunc ut imperdiet lobortis, magna neque tincidunt
lectus, id iaculis nisi justo id nibh. Pellentesque vel sem in erat
vulputate faucibus molestie ut lorem.

Quisque tristique urna in lorem laoreet at laoreet quam congue. Donec
dolor turpis, blandit non imperdiet aliquet, blandit et felis. In lorem
nisi, pretium sit amet vestibulum sed, tempus et sem. Proin non ante
turpis. Nulla imperdiet fringilla convallis. Vivamus vel bibendum nisl.
Pellentesque justo lectus, molestie vel luctus sed, lobortis in libero.
Nulla facilisi. Aliquam erat volutpat. Suspendisse vitae nunc nunc. Sed
aliquet est suscipit sapien rhoncus non adipiscing nibh consequat.
Aliquam metus urna, faucibus eu vulputate non, luctus eu justo.

Donec urna leo, vulputate vitae porta eu, vehicula blandit libero.
Phasellus eget massa et leo condimentum mollis. Nullam molestie, justo
at pellentesque vulputate, sapien velit ornare diam, nec gravida lacus
augue non diam. Integer mattis lacus id libero ultrices sit amet mollis
neque molestie. Integer ut leo eget mi volutpat congue. Vivamus sodales,
turpis id venenatis placerat, tellus purus adipiscing magna, eu aliquam
nibh dolor id nibh. Pellentesque habitant morbi tristique senectus et
netus et malesuada fames ac turpis egestas. Sed cursus convallis quam
nec vehicula. Sed vulputate neque eget odio fringilla ac sodales urna
feugiat.

Phasellus nisi quam, volutpat non ullamcorper eget, congue fringilla
leo. Cras et erat et nibh placerat commodo id ornare est. Nulla
facilisi. Aenean pulvinar scelerisque eros eget interdum. Nunc pulvinar
magna ut felis varius in hendrerit dolor accumsan. Nunc pellentesque
magna quis magna bibendum non laoreet erat tincidunt. Nulla facilisi.

Duis eget massa sem, gravida interdum ipsum. Nulla nunc nisl, hendrerit
sit amet commodo vel, varius id tellus. Lorem ipsum dolor sit amet,
consectetur adipiscing elit. Nunc ac dolor est. Suspendisse ultrices
tincidunt metus eget accumsan. Nullam facilisis, justo vitae convallis
sollicitudin, eros augue malesuada metus, nec sagittis diam nibh ut
sapien. Duis blandit lectus vitae lorem aliquam nec euismod nisi
volutpat. Vestibulum ornare dictum tortor, at faucibus justo tempor non.
Nulla facilisi. Cras non massa nunc, eget euismod purus. Nunc metus
ipsum, euismod a consectetur vel, hendrerit nec nunc.

\phantomsection\label{refs}
\begin{CSLReferences}{1}{0}
\bibitem[\citeproctext]{ref-Reference3}
Arnold, A. S., J. S. Wilson, and M. G. Boshier. 1998. {``A Simple
Extended-Cavity Diode Laser.''} \emph{Review of Scientific Instruments}
69 (3): 1236--39. \url{http://link.aip.org/link/?RSI/69/1236/1}.

\bibitem[\citeproctext]{ref-CTAN}
{``CTAN -- the Comprehensive TeX Archive Network.''} n.d. TeX Users
Group. Accessed March 15, 2024. \url{https://www.ctan.org}.

\bibitem[\citeproctext]{ref-Hao2023}
Hao, Example E. 2023. {``A Fictional Thesis on Advanced Fictional
Studies.''} PhD thesis, University of Examples.

\bibitem[\citeproctext]{ref-Reference1}
Hawthorn, C. J., K. P. Weber, and R. E. Scholten. 2001. {``Littrow
Configuration Tunable External Cavity Diode Laser with Fixed Direction
Output Beam.''} \emph{Review of Scientific Instruments} 72 (12):
4477--79. \url{http://link.aip.org/link/?RSI/72/4477/1}.

\bibitem[\citeproctext]{ref-Knuth1984}
Knuth, Donald E. 1984. \emph{The TeXbook}. Vol. A. Computers and
Typesetting. Reading, Massachusetts: Addison-Wesley.

\bibitem[\citeproctext]{ref-Lamport1986}
Lamport, Leslie. 1986. {``LaTeX: A Document Preparation System.''} In
\emph{User's Guide \& Reference Manual}. Reading, Massachusetts:
Addison-Wesley.

\bibitem[\citeproctext]{ref-Reference2}
Wieman, Carl E., and Leo Hollberg. 1991. {``Using Diode Lasers for
Atomic Physics.''} \emph{Review of Scientific Instruments} 62 (1):
1--20. \url{http://link.aip.org/link/?RSI/62/1/1}.

\end{CSLReferences}

\end{document}
