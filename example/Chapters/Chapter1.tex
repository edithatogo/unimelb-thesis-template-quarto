\chapter{Introduction}
\label{ch:introduction}

This chapter serves as an introduction to the thesis template and demonstrates various features, including citations, figures, tables, equations, and code listings.
The structure of this document is based on the University of Melbourne's guidelines for thesis preparation.
Key works in the field of document preparation include those by Knuth \cite{Knuth1984} and Lamport \cite{Lamport1986}.
We use \gls{latex} to typeset this document. We also use a \gls{svm}.
Further resources can be found on the Comprehensive TeX Archive Network (CTAN) \autocite{CTAN}.

\section{Citations and Bibliography}
\label{sec:citations}

This template uses `biblatex` for bibliography management, which offers great flexibility.
Here are some examples of citation commands:
\begin{itemize}
    \item \verb|\cite{key}|: Basic citation, e.g., \cite{Reference1}.
    \item \verb|\parencite{key}|: Parenthetical citation, e.g., \parencite{Reference2}.
    \item \verb|\textcite{key}|: Textual citation, e.g., \textcite{Reference3} showed that...
    \item \verb|\autocite{key}|: Context-aware citation, e.g., \autocite{Hao2023}.
    \item Multiple citations: \cite{Reference1, Reference2}.
    \item Citations with page numbers: See \cite[p.~15]{Knuth1984} for details.
    \item Another example: \textcite[see Chapter~3]{Lamport1986} provides an overview.
\end{itemize}
The bibliography is automatically generated at the end of the document. Make sure to run LaTeX, then Biber, then LaTeX twice more for all references to appear correctly. The `.latexmkrc` file in this template automates this process.

\section{Figures and Subfigures}
\label{sec:figures}

Figures are crucial in a thesis. This template uses the `graphicx` package for including images and `subcaption` for handling subfigures.
Always use vector graphics (like PDF or EPS) when possible for diagrams and plots to ensure scalability. For photographs, high-resolution raster graphics (PNG, JPG) are appropriate.
Remember to use `\graphicspath{{Figures/}}` in your main `.tex` file (as done in `example/Thesis.tex`) to specify the directory where your figures are stored.

\begin{figure}[htbp]
    \centering
    \begin{tikzpicture}
        \draw[thick,->] (0,0) -- (4,0) node[anchor=north west] {x};
        \draw[thick,->] (0,0) -- (0,4) node[anchor=south east] {y};
        \draw[scale=0.5,domain=-2:2,smooth,variable=\x,blue] plot ({\x},{\x*\x});
    \end{tikzpicture}
    \caption{Example of a figure created with `tikz`.}
    \label{fig:tikz_example}
\end{figure}

Figure \fref{fig:tikz_example} shows a simple figure. For more complex layouts, subfigures can be used, as shown in Figure \fref{fig:subfigures_example}.

\begin{figure}[htbp]
    \centering
    \begin{subfigure}[b]{0.45\textwidth}
        \centering
        \includegraphics[width=\textwidth]{unimelb_logo.eps} % Placeholder, replace with actual image
        \caption{First subfigure.}
        \label{fig:subfigure_a}
    \end{subfigure}
    \hfill % adds horizontal space between the subfigures
    \begin{subfigure}[b]{0.45\textwidth}
        \centering
        \includegraphics[width=\textwidth]{unimelb_logo.eps} % Placeholder, replace with actual image
        \caption{Second subfigure.}
        \label{fig:subfigure_b}
    </end{subfigure}
    \caption{Example of two subfigures using the `subcaption` package.}
    \label{fig:subfigures_example}
\end{figure}

Refer to subfigure \fref{fig:subfigure_a} or \fref{fig:subfigure_b} when discussing specific parts.

\section{Tables}
\label{sec:tables}

The `booktabs` package is included for creating professional-quality tables. It encourages avoiding vertical rules and using horizontal rules sparingly.
Table \tref{tab:example_table} provides an illustration.

\begin{table}[htbp]
    \centering
    \caption{An example table using `booktabs` and `siunitx`.}
    \label{tab:example_table_siunitx}
    \begin{tabular}{lS[table-format=3.2]S[table-format=2.1]}
        \toprule
        {Item} & {Value 1 (\si{\volt})} & {Value 2 (\si{\ampere})} \\
        \midrule
        Measurement 1 & 123.45 & 67.8 \\
        Measurement 2 & 1.23 & 45.6 \\
        Measurement 3 & 9.0 & 7.8 \\
        \bottomrule
    \end{tabular}
\end{table}

\section{Mathematical Equations}
\label{sec:equations}

The `amsmath` package provides numerous environments for typesetting mathematics.
Here's a simple numbered equation, Equation \eref{eq:einstein}:
\begin{equation}
    E = mc^2
    \label{eq:einstein}
\end{equation}
For a set of aligned equations, use the `align` environment, like in Equations \eref{eq:align_start} and \eref{eq:align_end}:
\begin{align}
    a + b &= c \label{eq:align_start} \\
    x + y + z &= w \label{eq:align_end}
\end{align}
The `gather` environment can be used for centering multiple equations:
\begin{gather}
    \int_0^\infty e^{-x^2} dx = \frac{\sqrt{\pi}}{2} \label{eq:gaussian_integral} \\
    \sum_{n=1}^\infty \frac{1}{n^2} = \frac{\pi^2}{6} \label{eq:basel_problem}
\end{gather}
Refer to \cref{ch:introduction} for more examples, or a specific \sref{sec:equations}.

\section{Code Listings}
\label{sec:code}

The `listings` package is used for typesetting code. Styles for Python and Matlab are predefined in `Thesis.cls`.
Listing \ref{lst:python_example} shows a Python example.

\begin{lstlisting}[language=Python, caption={A simple Python function.}, label={lst:python_example}]
def greet(name):
    """This function greets the person passed in as a parameter."""
    print(f"Hello, {name}!")

# Example usage
greet("World")
\end{lstlisting}

This concludes the demonstration of common thesis elements.